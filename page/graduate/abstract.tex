\cleardoublepage
\chapternonum{摘要}
目前船艇推进系统对高效低噪声推进泵的需求日益迫切,
在满足推进性能需求的基础上,
推进泵水下噪声已经成为一项关键技术指标。
为了更好的服务于推进泵低噪声设计,
有必要对推进泵噪声的声纹特征进行研究。
噪声的特征线谱是推进泵噪声研究中所关注的重点,
特征线谱不仅能表征推进泵的工作状态和结构信息,
也是声呐系统追踪和识别的重要信息,
但是其提取面临环境干扰强、直接测量难度大的问题。
因此,本文围绕推进泵噪声的声纹特征展开,针对噪声特征线谱提取的需求,
以紧凑型前置导叶的单级推进泵和新型结构的双级推进泵为研究对象,
设计噪声测试与分析系统并开展试验研究,
研究了推进泵噪声的声纹特征,
基于循环平稳分析方法实现了推进泵噪声的特征线谱提取。
本文主要研究内容包括以下几个方面:

(1)基于LabVIEW设计了推进泵噪声测试与分析系统。
其中涵盖了传感器、数据采集等硬件设计,以及信号分析、显示、存储等软件模块的设计。
该系统支持同步对多通道传感器信号实时采集,各通道信号同时分析、显示及存储,
具有操作简单、经济高效等优势。
基于系统的信号分析模块,可实现对噪声的频段能量分布特点、特征频段总声压量级、频谱特征等声纹特征分析。

(2)以紧凑型前置导叶的单级推进泵和新型结构的双级推进泵为研究对象,
在大型空泡水洞中对其分别开展了噪声试验。
在考虑背景噪声影响的基础上,
研究不同工况下推进泵噪声的声纹特征变化,
以及流速等与噪声的声学关联性。
结果表明,中低频段噪声对推进泵噪声有显著贡献。
噪声中低频段、高频段和全频段的总声压级随着流速的增大而增大,
流速变化对推进泵噪声能量分布和总声压级影响主要体现在中低频段。
两种形式的推进泵噪声频谱均表现为中低频线谱噪声、中低频宽带噪声和高频宽带噪声。
%频谱中包含丰富的线谱成分,能体现出轴频和叶频的谐波成分,
%但是这些特征频率在频谱中不能准确且快速的被识别。

(3)开展了推进泵噪声信号的特性研究。
结合推进泵噪声信号产生的机理和噪声信号特征,对噪声信号进行了
组分分析。推进泵噪声信号成分主要包括确定性信号分量、调制信号分量和环境噪声信号分量。
基于推进泵噪声信号的循环平稳特性和各组分的特点,
介绍了一阶和二阶循环平稳统计量,为噪声信号的预处理和调制信号的解调算法提供基础。
对匀速运转工况下的无空化噪声信号建立了幅值调制信号模型,
采用循环平稳分析方法对噪声信号的仿真模型进行了研究,
验证了循环平稳解调算法提取多组分调制频率的有效性和良好的抗噪性能。
 
(4)推进泵噪声的特征线谱提取和分析。针对噪声特征线谱提取的需求,
基于循环平稳分析方法实现了推进泵低频线谱的提取。
为了验证循环平稳分析方法提取特征线谱的可行性,采用 CFD 数
值模拟获得了推进泵脉动力特征,
结果表明循环平稳分析方法提取出的低频线谱成分可以和非定常脉动力的特征线谱对应起来。
同时,对推进泵噪声的调制特性进行了研究,
发现推进泵噪声信号具有强烈的调制特征,其中轴频和动叶叶频为主要的调制频率。
随着流速的增大,动叶叶频的调制效应也愈发显著。


\quad

关键词:推进泵;水洞噪声试验;声纹特征;信号处理;循环平稳;特征提取
%\cleardoublepage
%\chapternonum{Abstract}