\chapter{绪论}
\section{研究背景及意义}

近年来,喷水推进泵在高速船舶推进系统中得到了广泛应用,
随着动力传动系统与船体减振降噪技术的进步,推进泵噪声对舰船总体噪声的贡献度也被相对提高,
因而目前高新船舶对高效低噪声推进泵的需求日益迫切。
在满足推进性能需求的基础上,振动与声学特性目前已成为推进泵优化设计所关注的重点。
无论从推进泵低噪声设计或是从水声信号识别角度,对推进泵辐射噪声的特性与机理开展研究均具有重要意义。
推进泵噪声机理较为复杂,抛开空化噪声不谈仅从流致噪声角度考虑,其频谱已呈现宽带与线谱交叠的形貌,
其中线谱噪声对辐射噪声总级贡献度较大,很大程度上决定了推进泵的辐射噪声水平。
前期研究发现,推进泵叶轮与其他部件之间的动静干涉是重要的线谱噪声激励源,
动静干涉的抑制因此成为降低推进泵线谱振动与噪声的有效方法之一。


\section{研究现状}

我们可以用includegraphics来插入现有的jpg等格式的图片,
如\autoref{fig:zju-logo}所示。

\begin{figure}[htbp]
    \centering
    \includegraphics[width=.3\linewidth]{logo/zju}
    \caption{\label{fig:zju-logo}浙江大学LOGO}
\end{figure}


\subsection{推进泵振动噪声测试系统研究现状}
\subsection{推进泵噪声及流致振动激励源研究现状}
\subsection{循环平稳信号分析手段}


\par 如\autoref{tab:sample}所示,这是一张自动调节列宽的表格。

\begin{table}[htbp]
    \caption{\label{tab:sample}自动调节列宽的表格}
    \begin{tabularx}{\linewidth}{c|X<{\centering}}
        \hline
        第一列 & 第二列 \\ \hline
        xxx & xxx \\ \hline
        xxx & xxx \\ \hline
        xxx & xxx \\ \hline
    \end{tabularx}
\end{table}


\par 如\autoref{equ:sample},这是一个公式

\begin{equation}
    \label{equ:sample}
    A=\overbrace{(a+b+c)+\underbrace{i(d+e+f)}_{\text{虚数}}}^{\text{复数}}
\end{equation}

\section{研究内容和目标}
\subsection{研究内容}
\subsection{研究目标}

\chapter{振动噪声测试系统设计}

在噪声的测量中,采用1/3倍频程频谱分析能更加详细的反映出噪声源的频谱特性,
便于较全面的了解声源产生机理和提出最佳的降噪对策。

\section{系统总体设计概述}
\section{硬件系统设计}
\section{软件系统设计}
\section{系统校准}
\section{本章小结}

\chapter{推进泵噪声及流致振动激励特性试验研究}
\section{推进泵模型与测试试验台}
\section{推进泵压力脉动分析}
\section{推进泵噪声特性分析}
\section{推进泵流致振动激励特性分析}
\subsection{循环平稳信号分析理论}
\subsection{分析结果}
\section{本章小结}

\chapter{推进泵流致振动激励特性的改进研究}
\section{推进泵模型及数值计算方法}
\subsection{推进泵模型介绍}
\subsection{仿真方法及边界条件设置}
\section{试验结果分析}
\section{仿真结果分析}
\section{改进前后模型试验结果对比}
\section{本章小结}

\chapter{总结与展望}
\section{全文总结}
\section{创新点}
\section{展望}


\begin{figure}[htbp]
    \centering
    \includegraphics[width=.3\linewidth]{images.jpg}
    \caption{\label{fig:fig-placeholder}图片占位符}
\end{figure}