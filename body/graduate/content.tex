\chapter{绪论}
\section{研究背景及意义}

近年来,喷水推进泵在高速船舶推进系统中得到了广泛应用,
随着动力传动系统与船体减振降噪技术的进步,推进泵噪声对舰船总体噪声的贡献度也被相对提高,
因而目前高新船舶对高效低噪声推进泵的需求日益迫切。
在满足推进性能需求的基础上,振动与声学特性目前已成为推进泵优化设计所关注的重点。
无论从推进泵低噪声设计或是从水声信号识别角度,对推进泵辐射噪声的特性与机理开展研究均具有重要意义。
推进泵噪声机理较为复杂,抛开空化噪声不谈仅从流致噪声角度考虑,其频谱已呈现宽带与线谱交叠的形貌,
其中线谱噪声对辐射噪声总级贡献度较大,很大程度上决定了推进泵的辐射噪声水平。
前期研究发现,推进泵叶轮与其他部件之间的动静干涉是重要的线谱噪声激励源,
动静干涉的抑制因此成为降低推进泵线谱振动与噪声的有效方法之一。

(1)从调制与解调的基本概念出发,采用基于循环平稳信号分析手段对推进泵噪声信号进行解调研究,
重点分析了推进泵的噪声调制特征机理,得到了在进速系数下的解调谱。
(2)基于labVIEW平台开发了推进泵振动噪声测试系统,开展了振动及噪声信号频率和特征、信号特征频段等研究,
为进行推进泵振动噪声测量、判断推进泵声学性能及进行噪声及流致振动激励特性分析奠定了研究基础。 


\section{研究现状}

我们可以用includegraphics来插入现有的jpg等格式的图片,
如\autoref{fig:zju-logo}所示。

\begin{figure}[htbp]
    \centering
    \includegraphics[width=.3\linewidth]{logo/zju}
    \caption{\label{fig:zju-logo}浙江大学LOGO}
\end{figure}


\subsection{推进泵振动噪声测试系统研究现状}
\subsection{推进泵噪声及流致振动激励源研究现状}
\subsection{循环平稳信号分析手段}


\par 如\autoref{tab:sample}所示,这是一张自动调节列宽的表格。

\begin{table}[htbp]
    \caption{\label{tab:sample}自动调节列宽的表格}
    \begin{tabularx}{\linewidth}{c|X<{\centering}}
        \hline
        第一列 & 第二列 \\ \hline
        xxx & xxx \\ \hline
        xxx & xxx \\ \hline
        xxx & xxx \\ \hline
    \end{tabularx}
\end{table}
\section{研究内容和目标}
\subsection{研究内容}
\subsection{研究目标}

\chapter{振动噪声测试系统设计}

本文基于labVIEW平台开发了推进泵振动噪声测试系统,其中涵盖了信号采集的硬件系统设计,
以及实现信号分析和存储的软件系统设计,该系统适用于实现了振动与噪声信号的动态测量、实时采集、实时分析
和数据保存,同时也能够实现振动加速度传感器的性能检验和指标评估,适用于各类泵和风机振动及噪声测试场景。

基于此平台可开展对振动及噪声信号频率和特征、信号特征频段等研究,
为进行推进泵振动噪声测量、判断推进泵声学性能及进行噪声及流致振动激励特性分析奠定了研究基础。

\section{系统总体设计概述}

\subsection{测试基本参数和数据处理}
本文设计的测试信号对象包括声压和振动加速度信号,用来表征测试对象的声学和振动性能,可分别由麦克风或者水听器和
振动加速度传感器测量。声压级和振动加速度级是声学测量中常用的参数,一个声学量的级是该量与同类量的基准值之比的对数。
其中,声压级定义为将待测声压p$_e$与参考声压p$_{ref}$的比值取常用对数,再乘以20,以分贝计,即
\begin{equation}
    \label{equ:sample}
    L_{p} = 20\log_{10}{\left(p_{re}/p_{ref}\right )}
\end{equation}
式中,在水中基准声压为$p_{ref}= 1\times 10^{-6} \mathrm{P} a$,在空气中基准声压为$p_{ref}= 2\times 10^{-5} \mathrm{P} a$。
同理,振动加速度级也定义为加速度有效值a$_e$与基准加速度a$_{ref}$之比的以10为底的对数,再乘以20,以分贝计,即
\begin{equation}
    \label{equ:sample}
    L_{a} = 20\log_{10}{\left(a_{re}/a_{ref}\right )}
\end{equation}
式中,基准加速度值为$a_{ref}= 1\times 10^{-6} \mathrm{m/s^2} $。

信号的频谱是指信号的频率成分与能量分布的关系,可以体现信号的频率特征。同时在振动与噪声信号的分析中,
通常也会采用1/3倍频程频谱分析,是比较符合人耳分辨频率能力的频带划分方法,更加详细的反映噪声源的频谱特性。
通过将整个频谱划分为若干频带,每个频带的上限频率f$_u$与下限频率f$_l$之比是2的立方根,即满足以下公式:
\begin{equation}
    \label{equ:sample}
    f_{u}/f_{l}=2^{1/3}=1.2599
\end{equation}
中心频率f$_{m}$为上下限频率的几何平均值,即
\begin{equation}
    \label{equ:sample}
    f_{m}=\sqrt{f_{u}\cdot f_{l} } 
\end{equation}

便于较全面的了解声源产生机理和提出最佳的降噪对策。
\subsection{基本方案}
\section{硬件系统设计}
\section{软件系统设计}
\section{系统校准}
\section{本章小结}

\chapter{推进泵噪声及流致振动激励特性试验研究}
\section{推进泵模型与测试试验台}
\section{推进泵压力脉动分析}
\section{推进泵噪声特性分析}
\section{推进泵流致振动激励特性分析}
\subsection{循环平稳信号分析理论}
\subsection{分析结果}
\section{本章小结}

\chapter{推进泵流致振动激励特性的改进研究}
\section{推进泵模型及数值计算方法}
\subsection{推进泵模型介绍}
\subsection{仿真方法及边界条件设置}
\section{试验结果分析}
\section{仿真结果分析}
\section{改进前后模型试验结果对比}
\section{本章小结}

\chapter{总结与展望}
\section{全文总结}
\section{创新点}
\section{展望}


\begin{figure}[htbp]
    \centering
    \includegraphics[width=.3\linewidth]{images.jpg}
    \caption{\label{fig:fig-placeholder}图片占位符}
\end{figure}